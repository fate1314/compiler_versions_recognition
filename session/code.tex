
\section*{关键程序代码}
\vspace*{1cm}
\subsection*{汇编文件预处理}
\begin{lstlisting}
import os
import re

def process_assembly_file(input_file, output_file):
    # 读取原始汇编文件
    with open(input_file, 'r', encoding='utf-8') as file:
        lines = file.readlines()

    # 创建一个新的列表来存储修改后的内容
    processed_lines = []

    # 遍历原文件的每一行
    for line in lines:
        # 将逗号替换为空格
        line = line.replace(',', ' ')

        # 去除每行的前后空格
        stripped_line = line.strip()

        # 跳过以 . 开头或以 : 结尾的行
        if stripped_line.startswith('.') or stripped_line.endswith(':'):
            continue

        # 按空格分割行内的内容,得到每个单词
        words = re.split(r'\s+', stripped_line)

        # 处理每个单词
        processed_words = []
        for word in words:
            # 查找是否有括号,并替换为括号内的内容
            match = re.search(r'\((%[a-zA-Z0-9]+)\)', word)
            if match:
                word = match.group(1)

            # 处理以 $ 开头的单词
            if word.startswith('$'):
                if re.match(r'\$\d+', word):  # 如果 $ 后跟数字
                    word = '$0'
                elif re.match(r'\$-', word):  # 如果 $ 后跟负号
                    word = '$0'
                elif word.startswith('$.'):  # 如果 $ 后跟 .
                    word = '$.'
                elif word.startswith('$g'):  # 如果 $ 后跟 g
                    word = '$g'

            processed_words.append(word)

        # 将处理后的单词重新组合成一行
        processed_line = ' '.join(processed_words)
        processed_lines.append(processed_line + '\n')  # 保持行的格式

    # 写入新的汇编文件
    with open(output_file, 'w', encoding='utf-8') as file:
        file.writelines(processed_lines)


def process_all_files(input_folder, output_folder):
    # 如果输出文件夹不存在,则创建
    if not os.path.exists(output_folder):
        os.makedirs(output_folder)

    # 遍历输入文件夹中的所有文件
    for filename in os.listdir(input_folder):
        # 构造完整的文件路径
        input_file = os.path.join(input_folder, filename)
        output_file = os.path.join(output_folder, filename)

        # 处理每个文件
        process_assembly_file(input_file, output_file)


input_folders = ['8.4.0', '10.2.0', '11.3.0', '12.2.0', '13.2.0']
output_folders = [f"{folder}_processed" for folder in input_folders]

print(output_folders)

# 处理所有文件
for input_folder, output_folder in zip(input_folders, output_folders):
    process_all_files(input_folder, output_folder)

print("所有文件处理完毕。")

\end{lstlisting}
\vspace*{1cm}
\subsection*{特征提取}

\begin{lstlisting}
import csv
import os
from collections import Counter

file_count = 0  # 初始化文件计数器


def process_folder(folder_name):
    data = []
    global file_count
    file_count += 1  # 每处理一个文件夹,计数器加一
    for filename in os.listdir(folder_name):
        print(f"Processing {filename}...")
        if 'test' not in filename:
            print(f"Skipping {filename}...")
            continue
        file_path = os.path.join(folder_name, filename)
        with open(file_path, 'r') as file:
            word_counter = Counter()
            bigram_counter = Counter()
            first_word_list = []

            for line in file:
                words = line.strip().split()
                if not words:
                    continue
                first_word_list.append(words[0])
                word_counter.update(words)
                bigram_counter.update(zip(words[:-1], words[1:]))

            # 计算四元组(四个连续单词)的频率
            quadgram_counter = Counter(
                zip(first_word_list[:-3], first_word_list[1:-2],
                    first_word_list[2:-1], first_word_list[3:]))

            # 筛选频率大于10的特征
            word_features = {
                word: count
                for word, count in word_counter.items() if count > 10
            }
            bigram_features = {
                " ".join(bigram): count
                for bigram, count in bigram_counter.items() if count > 10
            }
            quadgram_features = {
                " ".join(quadgram): count
                for quadgram, count in quadgram_counter.items() if count > 10
            }

            # 构建一行数据
            features = {
                **word_features,
                **bigram_features,
                **quadgram_features
            }
            features['label'] = file_count  # 标签为当前文件的编号

            data.append(features)

    return data


def save_to_csv(data, output_file):
    keys = set()
    for row in data:
        keys.update(row.keys())

    keys = sorted(keys)  # 按字母排序,确保一致性
    with open(output_file, 'w', newline='') as csvfile:
        writer = csv.DictWriter(csvfile, fieldnames=keys)
        writer.writeheader()
        for row in data:
            writer.writerow(row)


# 定义已处理的文件夹列表
processed_folders = [
    '8.4.0_processed', '10.2.0_processed', '11.3.0_processed',
    '12.2.0_processed', '13.2.0_processed'
]

# 初始化用于保存所有文件夹数据的列表
all_data = []

# 遍历每个文件夹并处理数据
for folder in processed_folders:
    folder_data = process_folder(folder)
    all_data.extend(folder_data)

# 保存合并后的数据到CSV文件
save_to_csv(all_data, 'no_our_data.csv')

\end{lstlisting}
\vspace*{1cm}
\subsection*{模型建立}
\begin{lstlisting}
import pandas as pd
from sklearn.model_selection import train_test_split
import lightgbm as lgb
from sklearn.metrics import confusion_matrix, classification_report, ConfusionMatrixDisplay
import matplotlib.pyplot as plt
from sklearn.manifold import TSNE

df = pd.read_csv('processed_data2.csv')
df.fillna(0, inplace=True)
df[f'testb %al'], df['label'] = df['testb %al'], df['label']
X = df.drop(columns=['label'])
y = df['label']
y = y - 1

num_others_data = 1002
file1_idx = [i + num_others_data * j for i in range(1) for j in range(5)]
file2_idx = [i + num_others_data * j for i in range(1, 2) for j in range(5)]
our_data_idx = [i + num_others_data * j for i in range(2) for j in range(5)]
others_data_idx = df.index.difference(our_data_idx + file1_idx + file2_idx)
file1_data = X.iloc[file1_idx]
file2_data = X.iloc[file2_idx]
our_data = X.iloc[our_data_idx]

tsne = TSNE(n_components=2, perplexity=2, random_state=42, n_jobs=-1)
X_tsne = tsne.fit_transform(X)
df_tsne = pd.DataFrame(X_tsne, columns=['Component 1', 'Component 2'])
df_tsne['label'] = y
# 绘制 t-SNE 图像,使用标签 y 进行着色
plt.figure(figsize=(10, 10))
s = 30
scatter = plt.scatter(df_tsne['Component 1'],
                      df_tsne['Component 2'],
                      c=df_tsne['label'],
                      cmap='tab10',
                      marker='o',
                      edgecolor='k',
                      s=s)
plt.scatter(df_tsne.loc[file1_idx]['Component 1'],
            df_tsne.loc[file1_idx]['Component 2'],
            c='fuchsia',
            marker='o',
            edgecolor='k',
            label='file1',
            s=s)
plt.scatter(df_tsne.loc[file2_idx]['Component 1'],
            df_tsne.loc[file2_idx]['Component 2'],
            c='lightpink',
            marker='o',
            label='file2',
            edgecolor='k',
            s=s)

legend1 = plt.legend(*scatter.legend_elements(),
                     title="Labels",
                     loc='upper left')
plt.gca().add_artist(legend1)
plt.legend(loc='upper right')
plt.title('t-SNE Visualization')
plt.grid(True)
plt.show()

from xgboost import XGBClassifier
import warnings
from sklearn.model_selection import train_test_split

X_train, X_test, y_train, y_test = train_test_split(
    X,
    y,
    test_size=0.2,
)
warnings.filterwarnings('ignore')
xgb_params = {
    "objective": "binary:logistic",  # 目标函数:二分类逻辑回归
    "max_depth": 50,  # 树的最大深度
    "eta": 0.4,  # 学习率
    "subsample": 1,  # 采样率
    'sampling_method': 'uniform',  # 采样算法,uniform:均匀采样,gradient_based:梯度采样
    "colsample_bytree": 1,  # 构建每棵树时,使用特征的比例
    "colsample_bylevel": 1,  # 树的每一级的每一次分裂,对特征的采样比例
    "reg_lambda": 1,  # L2 正则化
    "reg_alpha": 1,  # L1 正则化
    "min_child_weight": 1,  # 最小的叶子节点样本权重和
    "gamma": 0,  # 节点分裂所需的最小损失函数减少量
    "eval_metric": "logloss",  # 评估指标:对数损失
    "scale_pos_weight": 1,  # 用于处理类别不平衡 negative:positive
    "tree_method": "hist",  # 树构建算法 ['exact', 'approx', 'hist', 'gpu_hist']
    'device': 'gpu',
    "nthread": -1,  # 使用所有可用线程
    # "seed":42,  # 随机数种子
    "n_estimators": 100,  # 决策树个数
    "booster": "gbtree",  # 基本模型
    'max_delta_step': 0,  # 限制每棵树权重改变的最大步长,0表示没有约束
}
model = XGBClassifier(**xgb_params)
model.fit(X_train, y_train)

y_pred = model.predict(X)
y_pred_ours = y_pred[file2_idx]
cm = confusion_matrix(y[file2_idx], y_pred_ours)
disp = ConfusionMatrixDisplay(confusion_matrix=cm)
disp.plot(cmap="Blues")
plt.title('Confusion Matrix')

y_pred = model.predict(X_test)
print(classification_report(y_test, y_pred))

\end{lstlisting}