\section{总结与展望}
编译器版本识别在现实中具有重要的应用价值。通过精确识别编译器的版本,能够帮助检测和预防与编译器相关的安全漏洞。例如,不同版本的编译器在处理某些指令集时可能存在特定的安全隐患,及时识别这些版本有助于安全团队采取针对性的防护措施,从而提升软件系统的整体安全性。

本文提出了一种基于操作码频率、寄存器使用率以及Bigram数量等特征的提取方法,用于区分和识别不同版本的编译器。通过对汇编代码的特征提取,能够捕捉到编译器版本之间的细微差异,使得模型在版本识别方面的准确性得到了显著提升。这种方法不仅为编译器版本的识别提供了新的技术路径,也为提高软件安全性提供了有力支持。该方法的应用可以扩展到各种编译器相关的安全分析场景,为开发和安全团队在应对复杂的安全威胁时提供了更加精细化的工具和手段。