\section{问题分析}
\subsection{问题一分析}
问题一要求我们提取出不同版本GCC编译附件1结果中的主要特征,以用于构建判别函数。为此,我们首先利用Vs Code的文本比较工具,了解各个汇编文件的差异,之后编写程序实现对多个版本编译的汇编文件进行特征提取。我们将考虑提取操作码和寄存器的频率、Bigram(连续两个指令组合)的数量、代码块数量等特征,这些特征能更全面地反映不同版本编译器的行为差异。通过对这些特征的分析,我们为后续的问题二提供了坚实的数据基础。
\vspace*{1cm}
\subsection{问题二分析}
在问题一的基础上,我们需要利用提取的特征数据构建判别函数。这是一个典型的多分类问题,特征数据将作为输入变量,不同的GCC版本作为标签类别。我们将使用机器学习算法,如Gini决策树、随机森林等,来构建判别模型。通过这些算法,我们能够实现不同版本GCC编译结果的有效分类和识别,并为后续的问题三提供强大的模型支持。
\vspace*{1cm}
\subsection{问题三分析}
问题三要求利用构建的判别器对不同版本GCC编译附件2的结果进行版本预测。在这个阶段,我们将模型应用于未见过的源代码2生成的汇编文件数据上,以验证其普适性和准确性。考虑到题目要求根据源代码1或2都能构建GCC版本判别函数,我们认为数据规模过小,难以构建具有泛化性的判别函数,所以选择增加数据集规模,以求构建具有高泛化性的判别函数。
\vspace*{1cm}
\subsection{问题四分析}
为了进一步提升判别函数的性能,我们提出了以下改进建议:首先,可以尝试引入深度学习模型,进行自动特征提取,再基于如Transformer的注意力机制模型,以捕捉更复杂的特征关系;其次,增加数据集的多样性,涵盖不同类型、不同规模的源代码,并通过不同的编译优化级别生成更多样的训练数据;最后,通过结合静态和动态特征,使模型能够从多个维度进行学习和预测,从而提高模型的泛化能力和预测准确率。
\vspace*{1cm}