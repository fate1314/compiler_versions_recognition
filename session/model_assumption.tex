\section{模型假设}
\begin{enumerate}%创建有序列表
  \item 假设不同版本编译器对于实现同一功能,偏向于使用某些特定的操作码与寄存器。因为随着硬件的进步,指令更新,编译器将采用新的指令以实现更快的功能。
  \item 假设采用Windows下的mingw64进行编译源代码。因为附件1中的源代码含有easyx图形库,导致gcc无法在Linux系统下编译代码。所以我们考虑到既要能够编译,也要使用与gcc相关的编译器,故使用mingw64。
  \item 假设不提取汇编文件“.ident”的相关特征。无论是gcc还是mingw64采用默认编译出来的汇编文件,都含有“.ident”这一项,该项会直接注明汇编文件是由哪个版本的编译器生成的。我们考虑到实际应用中需要判别的是二进制文件,通过二进制文件反编译成汇编文件会失去'.ident'这个特征。
\end{enumerate}