\begin{abstract}
    编译器是一种将高级编程语言的源代码转换为机器语言的工具,使计算机能够理解并执行这些指令。在软件开发中,它扮演着至关重要的角色,不仅提高了程序的可读性和可维护性,还能优化程序的执行效率。然而,不同版本的编译器在优化策略、支持的语言特性以及生成的机器代码方面可能存在显著差异,这些差异会影响程序的性能、兼容性和安全性。通过识别编译器版本,开发者可以更好地理解和控制程序的行为,确保软件在不同环境中的稳定性和一致性。此外,编译器版本的识别在软件逆向工程、安全分析和版权保护等领域也具有重要的应用价值。

针对问题一:我们首先选定了GCC的‘8.4.0’、‘10.2.0’、‘11.3.0’、‘12.2.0’和‘13.2.0’五个版本作为识别任务的目标。接着,利用Vs Code的文本比较工具对五个汇编文件进行了两两比较,人工分析各汇编文件之间的差异,随后编写程序提取各版本编译的汇编文件的特征。我们提取了操作码、寄存器、Bigram(连续两个指令组合)的数量、代码块数量等特征,这些特征能较全面地反映不同版本编译器的行为差异。对这些特征的分析为后续的问题二提供了坚实的数据基础。

针对问题二:我们首先进行了数据分析和数据可视化,了解数据的基本性质和分布特征。基于数据分析和文献查阅,我们决定选择一个非线性模型来实现编译器版本识别。经过综合考量,我们决定使用机器学习中的Gini决策树算法,使用提取的特征作为输入变量,不同的GCC版本作为标签类别,成功构建了一颗Gini决策树作为判别函数,为之后的问题解决提供了模型基础。

针对问题三:我们同样使用问题一中的特征提取程序,对源代码2生成的五个汇编文件进行特征提取,并使用问题二构建的Gini决策树进行模型预测。最终,模型的预测准确率为60\%。尽管模型效果尚可,但由于数据集的不足,Gini决策树可能存在欠拟合问题,泛化能力较低。因此,我们选择使用CSmith增加数据集,并使用集成模型XGBoost进行判别函数的构建。

针对问题四:我们提出了多项模型改进建议,包括引入深度学习技术进行自动特征提取、增加数据集的多样性,以及结合静态和动态特征来提升模型的泛化能力。我们希望通过实现这些改进,能够显著提升编译器版本识别模型的准确性和鲁棒性。
\end{abstract}
