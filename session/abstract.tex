\begin{abstract}
    编译器是将高级编程语言编写的源代码转换为机器语言的工具,使计算机能够理解和执行这些指令。它在软件开发中扮演着关键角色,不仅提高了程序的可读性和可维护性,还优化了程序的执行效率。然而不同版本的编译器在优化策略、支持的语言特性和生成的机器代码上可能存在显著差异,这会影响程序的性能、兼容性和安全性。通过识别编译器版本,开发者可以更好地理解和控制程序的行为,确保软件在不同环境中的稳定性和一致性。此外,识别编译器版本对于软件逆向工程、安全分析和版权保护等领域也具有重要的应用价值。
    \par
    本文提出了一种基于机器学习的编译器版本识别方法,通过提取不同版本的编译器的编译结果特征,包括操作码,操作数,n-gram,block频率,我们使用机器学习模型构建了一个判别函数,用于识别给定汇编文件的编译器版本。
\end{abstract}