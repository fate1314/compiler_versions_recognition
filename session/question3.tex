
\section{问题3}
\subsection{模型测试}
按照问题一中给出的提取特征的步骤,我们对程序源代码2生成的五个汇编文件进行特征提取。提取的特征数据如\autoref{tab:2}所示。
\begin{table}[H]
\caption{\textbf{源代码2特征数据}}%标题
\label{tab:2}
\centering%把表居中
\begin{tabular}{lccccc}%三列,内容全部居中
\toprule%第一道横线
 feature\_name&8.4.0 & 10.2.0 & 11.3.0 & 12.2.0 & 13.2.0 \\
\midrule%第二道横线
\$0\_\%ecx&3&3&3&3&2 \\
\$0\_\%rax&4 & 85 & 85 & 4 & 4 \\
\$0\_\%rbp&40 & 40 & 40 & 40 & 38 \\
\$0\_\%rsp&16 & 16 & 16 & 16 & 10 \\
\%eax & 242 & 375 & 375 & 242&242 \\
...&... & ... & ... & ... & ... \\
salq\_addq\_leaq&81 & 52 & 52 & 81 & 81 \\
subq&8 & 8 & 8 & 8 & 5 \\
subq\_\$0&8 & 8 & 8 & 8 & 5 \\
subq\_leaq\_movq&1 & 1 & 2 & 2 & 1 \\
subq\_movl\_movl&3 & 3 & 3 & 3 & 1 \\
\bottomrule%第三道横线
\end{tabular}
\end{table}

我们直接使用问题二构建的Gini决策树进行版本预测,最终得到的预测结果如\autoref{fig:tree_pre}所示:
\begin{figure}[H]
    \centering
    % \includegraphics[width=0.75\linewidth]{figures/tree_pre.png}
    \caption{问题二判别函数预测结果}
    \label{fig:tree_pre}
\end{figure}
在这张图中,标签 0, 1, 2, 3, 4 分别对应的是 GCC 编译器版本 8.4.0、10.2.0、11.3.0、12.2.0 和 13.2.0。从图像中我们可以清楚地看到,对于真实标签为 1, 2, 3 的汇编文件,模型全部预测正确,说明模型在这些版本之间的区分能力较好。然而,对于真实标签为 0 和 4 的汇编文件,模型却将它们全部错误地预测为标签 3 的汇编文件。这种明显的误分类现象表明,模型在区分这些特定版本时存在严重问题。

考虑到当前数据集的样本量相对较少,以及决策树模型在处理数据复杂性和高维特征时的局限性,我们的判别函数存在显著的不足。因此,模型的预测性能需要进一步改进。
\vspace*{1cm}

\subsection{决策树模型分析}
决策树模型作为一种简单且直观的机器学习算法,广泛应用于分类和回归任务。然而,尽管其易于理解和解释,决策树模型也存在一些固有的缺陷,决策树模型存在以下几个缺陷:
\begin{itemize}
	\item 容易过拟合:决策树模型在处理复杂数据集时,容易出现过拟合问题。这是因为决策树模型在构建过程中会根据数据的分布来选择分裂点,如果数据集过于复杂,决策树可能会生成\textbf{过深的树结构},从而导致模型过拟合,泛化能力较差。

	      \tbox{\autoref{fig:fit_status}展示了不同拟合状态的大致情况,过拟合是模型完全拟合了训练数据,把训练的数据的噪声也当做数据的统计特征,从而导致在训练集的误差很低,但是在测试集的误差很高。}
	\item 对数据噪声敏感:决策树对数据中的噪声和异常值非常敏感。这是因为决策树在构建过程中会根据数据的分布来选择分裂点,噪声和异常值可能导致树的结构发生较大变化,从而影响模型的稳定性和准确性。
	\item 不稳定:由于决策树的构建过程是基于局部最优的贪心算法,因此对数据的变化非常敏感,小的输入数据变化可能导致决策树结构的显著变化。
	\item 计算复杂度:在特征数量较多或数据集较大时,决策树的构建过程可能会非常耗时。这是因为每次分裂都需要计算所有可能分裂点的增益,从而选择最佳分裂点。
\end{itemize}
\begin{figure}[H]
	\centering
	\includegraphics[width=0.8\textwidth]{figures/fit_status.png}
	\caption{不同拟合状态}
	\label{fig:fit_status}
\end{figure}

并且,在我们的实验中,我们发现如果只使用附件1或者是附件2来构建模型,模型的预测结果将会趋向于随机猜测,而非根据编译器的特征进行预测。
\par
从数据的分布来看,我们使用t-SNE算法对数据进行降维可视化,t-分布邻域嵌入(t-distributed Stochastic Neighbor Embedding,t-SNE)\cite{JMLR:v9:vandermaaten08a}是一种非线性降维技术,特别适用于高维数据的可视化。t-SNE通过将高维数据映射到低维空间,同时\textbf{尽可能保持数据点之间的局部结构},从而揭示数据的内在结构和模式。t-ESN算法通过优化KL散度来最小化高维空间和低维空间之间的分布差异,从而实现降维。t-SNE的优化目标是最小化高维空间和低维空间之间的KL散度,KL散度定义如下:
\begin{equation}
	C=KL(P||Q)=\sum_i\sum_jp_{ij}\log\frac{p_{ij}}{q_{ij}},
\end{equation}
其中,$p_{ij}$是高维数据点$x_i$和$x_j$之间的相似度,在高纬空间中,t-SNE算法使用高斯分布来衡量高维空间中点对之间的相似性。$q_{ij}$是低维数据点$y_i$和$y_j$之间的相似度。在低维空间中,t-SNE使用学生t分布(t-distribution)来计算点对之间的相似性。这个分布的长尾特性使得它能够更好地捕捉低维空间中的局部结构。t-SNE通过梯度下降算法优化该目标,C对于低维数据的梯度为
\begin{equation}
	\frac{\delta C}{\delta y_i}=4\sum_j(p_{ij}-q_{ij})(y_i-y_j)\left(1+\|y_i-y_j\|^2\right)^{-1},
\end{equation}
最后使用梯度更新低维数据点$y_i$
\begin{equation}
	y_i^{(t+1)}=y_i^{(t)}+\eta\frac{\delta C}{\delta y_i},
\end{equation}
其中$\eta$是学习率。在t-SNE算法中,还引入了\textbf{动量}来进一步优化低位数据点的表示:
\begin{equation}
	momentum=\alpha(y_i^{(t)}-y_i^{(t-1)}).
\end{equation}
动量项通过在更新过程中加入前几次更新的累积效果,可以帮助算法在梯度方向上更快地移动,并且通过平滑更新路径,能够有效减少震荡,使得优化过程更加稳定。加入动量后的更新公式为
\begin{equation}
	y_i^{(t+1)}=y_i^{(t)}+\eta\frac{\delta C}{\delta y_i}+\alpha(y_i^{(t)}-y_i^{(t-1)})
\end{equation}
综上所述,t-SNE算法流程如\autoref{alg:tsne}所示。
\begin{breakablealgorithm}
	\caption{t-SNE}
	\label{alg:tsne}
	\begin{algorithmic}
		\REQUIRE 数据集$X$,目标维度$k$,,优化参数:迭代次数$T$,学习率$\eta$,动量参数$\alpha$
		\ENSURE  低维数据$\mathcal{Y}$

		\STATE 计算高维数据之间的相似度$p_{j|
					i}$
		\[
		p_{j|i}=\frac{\exp\left(-\|x_i-x_j\|^2/2\sigma_i^2\right)}{\sum_{k\neq i}\exp\left(-\|x_i-x_k\|^2/2\sigma_i^2\right)}
		\]
		其中$\sigma_i$是以$\boldsymbol{x_i}$为中心的高斯分布的方差,通过二分搜索确定
		\STATE 从$\mathcal{N}(0,10^{-4}I)$从采样$\mathcal{Y}^{(0)}={y_1,y_2,\cdots,y_n}$
		\FOR {$i=1$ to $T$}
		\STATE 计算低维数据之间的相似度$q_{j|i}$
		\[q_{j|i}=\frac{(1+\|y_i-y_j\|^2)^{-1}}{\sum_{k\neq i}(1+\|y_i-y_k\|^2)^{-1}}\]
		\STATE 计算梯度
		\[
			\frac{\delta C}{\delta y_i}=4\sum_j(p_{ij}-q_{ij})(y_i-y_j)\left(1+\|y_i-y_j\|^2\right)^{-1}.
		\]
		\STATE 更新$y_i$
		\[
			y_i^{(t+1)}=y_i^{(t)}+\eta\frac{\delta C}{\delta y_i}+\alpha(y_i^{(t)}-y_i^{(t-1)})
		\]
		\ENDFOR
	\end{algorithmic}
\end{breakablealgorithm}
我们使用Scikit-learn\cite{scikit-learn}实现t-SNE算法的可视化,结果如\autoref{fig:tsne}所示。
\begin{figure}[H]
	\centering
	\includegraphics[width=0.7\textwidth]{figures/tsne.png}
	\caption{t-SNE降维可视化}
	\label{fig:tsne}
\end{figure}
可以发现,附件1和附件2在整体上的数据分布相差很远,并且在局部上同一个源文件的不同版本之间也没有明显的分布规律。这也是导致决策树模型无法准确预测的原因之一。
\vspace{1cm}
\subsection{模型改进}
通过对决策树模型和数据分布的分析,我们发现以下问题
\begin{itemize}
	\item \textbf{训练数据不具备广泛性}:原有的训练数据少,并且类型单一,无法体现编译器在编译技术上的优化方法和特点
	\item \textbf{模型本身的缺陷}:单纯的决策树模型容易过拟合:在没有剪枝方法的情况下,决策树的深度容易过深,导致模型过拟合,泛化能力较差。但是如果使用剪枝方法,则决策树又容易欠拟合,导致模型的准确率较低
\end{itemize}
\par
针对训练的问题,我们需要根据不同版本编译器的特点来对应生成数据。使用对应生成的数据训练模型,可以使得模型能够捕捉到不同版本编译器在编译技术和优化方法上的差异。我们注意到一个软件:\textbf{CSmith}。该软件是用于编译器压力测试的随机程序生成器。它专门用于生成程序源代码,以帮助测试和验证编译器的正确性和稳健性。生成的代码有以下特点:
\begin{itemize}
	\item 1.CSmith生成的代码涵盖了编程语言的大部分特性,包括指针、数组、结构体以及复杂的控制流结构等。这种广泛的覆盖性有助于发现不同版本编译器在编译技术上的差异。
	\item 2.随机性和多样性:生成的代码是随机的,这意味着每次运行CSmith都会产生不同的程序。这进一步提高生成的代码的随机性和多样性。
\end{itemize}
基于CSmith,我们生成了1000个随机程序,并使用不同版本的编译器对这些程序进行编译,得到了1000个数据样本。
\par
由于我们引入了大量数据,并且决策树模型存在固有的缺陷,因此我们考虑使用\textbf{集成模型}来构建判别函数。集成模型是一种强大的机器学习方法,首先构建多个弱分类器,然后在把这些弱分类器的分类结果进行加权,得到一个强分类器。\autoref{fig:ensemble}显示了集成模型的基本原理。
\begin{figure}
	\centering
	\includegraphics[width=0.7\textwidth]{figures/ensemble.png}
	\caption{集成模型的基本原理}
	\label{fig:ensemble}
\end{figure}
在选择机器学习模型时,我们主要考虑以下几个关键特性:\textbf{训练速度快、能够处理高维数据以及具备良好的泛化能力}。基于这些特性,我们选择了\textbf{XGBoost}模型。\textbf{XGBoost(Extreme Gradient Boosting)}\cite{10.1145/2939672.2939785}是一种高效且灵活的梯度提升框架,通过集成决策树模型并逐步优化,展现出卓越的性能和可扩展性,尤其在处理大规模数据集时表现优异。XGBoost的主要优点包括:
\begin{itemize}
	\item 高性能:XGBoost通过直方图优化、GPU加速以及数据/特征并行等技术,大幅提升了训练效率,相较于一般的集成模型,它能够更快速地完成训练过程。
	\item 灵活性:作为一个集成决策树模型,XGBoost不仅能够自动处理缺失值,还能处理数值型和类别型等多种类型的数据。这种灵活性使其非常契合我们数据集的特点。
	\item 准确性:XGBoost通过拟合决策树的预测残差,逐步优化树模型的结构和权重,从而提升预测准确性。这种逐步优化的方式有效地提高了模型的整体性能。
	\item 可解释性:XGBoost提供了特征重要性评估和模型解释功能,使我们能够更直观地理解模型的预测结果,以及各特征对预测结果的影响。这种可解释性对于模型的应用和改进具有重要意义。
\end{itemize}
通过这些优势,XGBoost成为我们在处理复杂数据集和追求高效、准确预测时的理想选择。
XGBoost的原理如下:
假设已经训练并且集成了$t-1$个模型:
\begin{equation}
	F_t=\sum_{i=1}^{t-1}f_i(x),
\end{equation}
现在准备增加一个模型$f_t(x)$,并集成到模型中,使得目标函数:
\begin{equation}
	\mathcal{L}(F_{t-1}+\alpha f_t)=\sum_{i=1}^{n}l(y_i,\hat{y}^{(t)})+\Omega(f_t),
\end{equation}
最小化,其中$\hat{y}^{(t)}=F_{t-1}(x)+f_t(x)$,$\Omega(f_t)$是正则化项,$l(y_i,\hat{y}^{(t)})$是损失函数,$n$ 是样本数量。我们对损失函数$l(y_i,\hat{y}^{(t)})$在$F_{t-1}(x)+f_t(x)$处进行二阶泰勒展开,并且忽略余项,那么我们可以得到 \begin{equation}
	l(y_i,\hat{y}^{(t)})\approx\sum_{i=1}^{n}\left[l\left(y_{i},\hat{y}_{i}^{(t-1)}\right)+g_{i}f_{t}(x_{i})+\frac{1}{2}h_{i}f_{t}^{2}(x_{i})\right],
\end{equation}
其中$g_i=\nabla_{\hat{y}^{(t-1)}}l\left(y_i,\hat{y}_i^{(t-1)}\right)\quad h_i=\nabla_{\hat{y}^{(t-1)}}^2l\left(y_i,\hat{y}_i^{(t-1)}\right)$,对于每个叶子结点的预测
\begin{equation}
	f_t(x)=w_{q(x)},w\in\mathbb{R}^T,q\colon\mathbb{R}^d\mapsto\{1,2,...,T\},,,,,
\end{equation}
其中$T$是叶子结点的个数,$q(x)$是样本$x$在树中的叶子结点的索引(因为每个样本最终都会落到某个叶子节点上),$w$是叶子结点的权重,那么我们就可以定义正则化函数:
\begin{equation}
	\Omega(f_t)=\gamma T+\frac{1}{2}\lambda\sum_{i=1}^{T}w_i^2,
\end{equation}
$\gamma$和$\lambda$是超参数,分别控制叶子结点的个数和叶子结点的惩罚权重,最终的优化目标为
\begin{equation}
	\begin{aligned}
		\mathcal{L}^{(t)} & \simeq\sum_{i=1}^{n}\left[l\left(y_{i},\hat{y}_{i}^{(t-1)}\right)+g_{i}f_{t}(x_{i})+\frac{1}{2}h_{i}f_{t}^{2}(x_{i})\right]+\Omega(f_{t})
		\\
		                  & =\sum_{i=1}^{n}\left[g_{i}f_{t}(x_{i})+\frac{1}{2}h_{i}f_{t}^{2}(x_{i})\right]+\gamma T+\frac{1}{2}\lambda\sum_{j=1}^{T}w_{j}^{2}+const
		\\
		                  & =\sum_{j=1}^{T}\left[(\sum_{i\in I_{i}}g_{i})w_{j}+\frac{1}{2}(\sum_{i\in I_{i}}h_{i}+\lambda)w_{j}^{2}\right]+\gamma T+const
	\end{aligned}
\end{equation}
其中$I_j$是样本集合$\{i|q(x_i)=j\}$,也就是由于每个样本最终都会落到某个叶子结点上,因此把对样本的求和转换为对叶子结点的求和,并且$l(y_i,y^{(t-1)})$是已经确定的常数,因此最终优化目标是
\begin{equation}
	\mathcal{L}^{(t)}=\sum_{j=1}^{T}\left[(\sum_{i\in I_{i}}g_{i})w_{j}+\frac{1}{2}(\sum_{i\in I_{i}}h_{i}+\lambda)w_{j}^{2}\right]+\gamma T
\end{equation}
这是一个关于$w_i$的二次函数,可以通过求导得到最优解,令
\begin{equation}
	G_j=\sum_{i \in I_j}g_i\quad H_j=\sum_{i \in I_j}h_i,
\end{equation}
则优化目标的闭式解为
\begin{equation}
	w_{j}^{*}=-\frac{G_{j}}{H_{j}+\lambda}\quad \mathcal{L}^{(t)}=-\frac{1}{2}\sum_{i=1}^{T}\:\frac{G_{j}^{2}}{H_{j}+\lambda}+\gamma T,
\end{equation}
那么在构建决策树时,与之前的决策树一样,我们可以通过贪心算法,从根节点开始,对每个叶子结点,计算其最优的权重$w_j$,然后计算损失函数,
\begin{equation}
	\text{Gain}=\frac{1}{2}(\frac{G_{L}^{2}}{H_{L}+\lambda}+\frac{G_{R}^{2}}{H_{R}+\lambda}-\frac{(G_{L}+G_{R})^{2}}{H_{L}+H_{R}+\lambda})-\gamma,
\end{equation}
其中$G_L,H_L$和$G_R,H_R$分别是左右子树的梯度值,最后选择最优的划分特征和划分点,递归构建树,直到满足停止条件。
在本题中,由于是分类问题,我们使用交叉熵作为损失函数,
\begin{equation}
	l(y_i,\hat{y}_i^{(t-1)})=-\left(y_i\log(\hat{y}_i^{(t-1)})+(1-y_i)\log(1-\hat{y}_i^{(t-1)})\right)
\end{equation}
则该损失函数的梯度和二阶导数分别为
\begin{equation}
	g_i=\hat{y}_i^{(t-1)}-y_i ,\quad h_i=\hat{y}_i^{(t-1)}(1-\hat{y}_i^{(t-1)}),
\end{equation}
当递归构建完成一棵树后,就可以把新构建的树集成到模型中:
\begin{equation}
	F_t=F_{t-1}+\eta f_t,
\end{equation}
其中$\eta$是超参数学习率.最终,XGBoost的算法如\autoref{alg:xgboost}所示。
\begin{breakablealgorithm}
	\caption{XGBoost}
	\label{alg:xgboost}
	\begin{algorithmic}[1]
		\REQUIRE 超参数:$\lambda$ 和 $\gamma$。特征$X$,标签$Y$。停止条件:树的最大深度$K$,集成个数M。
		\ENSURE 集成模型XGBoost

		\STATE 初始化模型的预测值 $\hat{y}_i^{(0)} = 0$ 对于所有样本 $i$
		\FOR{每轮迭代 $t = 1, 2, \ldots, M$}
		    \STATE \textbf{计算梯度和二阶导数}:
		    \FOR{每个样本 $i$}
		        \STATE 计算损失函数的一阶梯度 $g_i = \nabla_{\hat{y}_i^{(t-1)}} l(y_i, \hat{y}_i^{(t-1)})$ 和二阶导数 $h_i = \nabla_{\hat{y}_i^{(t-1)}}^2 l(y_i, \hat{y}_i^{(t-1)})$。
		    \ENDFOR
		    \STATE \textbf{遍历所有特征}:
		    \FOR{每个特征 $j$}
		        \STATE 计算所有可能的分裂点。
		        \FOR{每个分裂点}
		            \STATE 将数据集分为左子集 $L$ 和右子集 $R$。
		            \STATE 计算左子集和右子集的梯度和二阶导数之和:
		            \STATE $G_L = \sum_{i \in L} g_i, \quad H_L = \sum_{i \in L} h_i$
		            \STATE $G_R = \sum_{i \in R} g_i, \quad H_R = \sum_{i \in R} h_i$
		            \STATE 计算分裂增益:
		            \STATE $\text{Gain} = \frac{1}{2}\left(\frac{G_L^2}{H_L + \lambda} + \frac{G_R^2}{H_R + \lambda} - \frac{(G_L + G_R)^2}{H_L + H_R + \lambda}\right) - \gamma$
		        \ENDFOR
		    \ENDFOR
		    \STATE \textbf{选择最佳分裂点}:
		    \STATE 对于每个特征,选择具有最大增益的分裂点。
		    \STATE 在所有特征中,选择增益最大的分裂点作为最终的分裂点。
		    \STATE \textbf{递归分裂}:
		    \STATE 对于新的左子树和右子树,重复步骤3至7,直到满足停止条件。
		    \STATE \textbf{更新模型}:
		    \STATE 使用当前树的预测结果更新模型的预测值 $\hat{y}_i^{(t)} = \hat{y}_i^{(t-1)} + \eta f_t(x_i)$,其中 $\eta$ 是学习率,$f_t(x_i)$ 是当前树的预测。
			\STATE 更新模型
		     \[
			 F_t=F_{t-1}+\eta f_t(x)
			 \]
		\ENDFOR
		\RETURN
	\end{algorithmic}
\end{breakablealgorithm}
\par
对于$d$个特征,每个特征,都可以以$O(1)$的速度构建计算增益,每一层都需要$n\log n$的时间排序选出最佳特征,一共$K$层。一共有$M$棵树,因此整个XGBoost算法的时间复杂度为$O(MKdn\log{n})$
\vspace*{1cm}
\subsection{实验结果}
基于XGBoost,超参数如下:树的最大深度max\_depth=50 ;学习率$\eta$=0.4 ;正则化参数$\lambda$=1 ;叶子结点权重的正则化参数$\gamma$=0,集成模型个数n\_estimators=100,基本模型为\textbf{gbtree}。我们使用1000个随机生成的程序样本和附件一进行训练,得到的模型在附件二的预测结果如\autoref{fig:xgboost}所示。
\begin{figure}[H]
	\centering
	\includegraphics[width=0.7\textwidth]{figures/new_corr.png}
	\caption{XGBoost预测结果}
	\label{fig:xgboost}
\end{figure}
从\autoref{fig:xgboost}可以看出,XGBoost预测准确率为\textbf{0.6},相较于一般的决策树模型提高了一倍。