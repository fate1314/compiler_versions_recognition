\section{问题四}
提高由编译结果区分编译器版本的判别函数性能的建议:
\vspace*{1cm}
\subsection{自动特征提取}
特征工程将原始数据转化为更适合模型学习的特征的过程,直接影响模型的性能和准确性。然而,传统的手动特征工程存在一些缺陷。首先,手动特征工程依赖于领域专家的知识和经验,这将导致特征选择的主观性和局限性。其次,手动提取特征通常是一个耗时且繁琐的过程,尤其在高维数据或复杂数据结构的情况下,难以保证全面和高效。因此,如果要提高模型的区分度和性能,我们可以考虑采用深度学习技术来进行自动特征提取我们构想了一个基于协同注意力机制的模型。具体来说,首先将源文件和对应的汇编文件转换为词向量嵌入表示。然后,利用基于 Transformer 的自注意力机制来捕捉文件内部的语义特征。为了更好地理解源文件与汇编文件之间的关系,进一步采用 Cross-Attention 机制来提取它们之间的语义相似性和结构相异性。这种方法的优势在于,它能够自动捕捉不同编译器版本之间在语义和结构上的细微差异,这些差异可以作为有效的特征,用于构建高效的分类模型。
\vspace{1cm}
\subsection{增加数据集多样性}
考虑到不同领域的源代码会存在很大的差异,例如同为C++程序代码,后端服务器代码往往倾向于网络和IO操作,而游戏的代码则会包含到大量的图形渲染和物理计算,因此不同领域的源代码在同一个版本的编译器编译的编译结果存在很大差异。此外,即使是同一个领域的源代码,不同规模的源代码之间也会存在非编译器造成的差异。最后编译器版本之间的差距主要是体现在对于不同版本的语言支持以及静态代码优化技术,而后者一般要通过编译指令开启代码优化才能体现。鉴于以上分析,我们可以通过以下几种方式来增加数据集的多样性,提高模型的泛化能力:
\begin{itemize}
    \item 不同类型的程序:收集多种类型的程序(如计算密集型、IO 密集型、混合型)进
    行编译,提取汇编结果。
    \item 不同规模的程序:从小规模到大规模的程序进行编译,确保模型能泛化到不同复
    杂度的代码。
    \item 编译优化级别:使用不同的优化级别(如-O0,-O1,-O2,-O3)进行编译,捕捉
    编译器在不同优化下的行为差异。
\end{itemize}
\vspace*{1cm}
\subsection{结合静态和动态特征}
除了源代码和汇编代码之外,我们还可以考虑结合静态特征和动态特征来提高模型的区分度。具体来说,我们可以提取以下两类特征:
\begin{itemize}
    \item 静态特征:提取静态汇编特征,如指令频率、寄存器使用率、基本块大小。
    \item 动态特征:运行程序,收集运行时的行为特征,如执行路径、内存访问模式、CPU
    使用率。
 我们可以通过结合静态和动态的特征进行分析,更全面地了解编译器版本之间的差异,提高模型的泛化能力。
\end{itemize}
