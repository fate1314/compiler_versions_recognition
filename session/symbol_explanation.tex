\renewcommand{\arraystretch}{1.5}
\section{符号说明}
\begin{table}[H]
\caption{\textbf{符号说明}}%标题
\centering%把表居中
\begin{tabular}{cc}%三列,内容全部居中
\toprule%第一道横线
 符号 &说明\\ 
\midrule%第二道横线 
 D & 给定样本集合 \\
$p_i$ & 样本集中属于第i类的样本比例 \\
  C& 类的数量 \\
  $|D_{left}|$ & 划分后数据集$D_left$中样本的数量 \\
   $|D_{right}|$ & 划分后数据集$D_right$中样本的数量 \\
    $|D|$ & 数据集$D$中总样本的数量 \\
    $D_{not-missing}$&属性A有值的样本集\\
    $D_{missing}$&属性A有缺失值的样本集\\
     $Gini(D)$ & 数据集D的基尼指数 \\
     $Gini(D,A,t_i)$&数据集D经特征A的划分点划分后的基尼指数\\
     $T$&叶子节点个数\\
     $q(x)$&样本x在书中的叶子节点的索引\\
     $\omega$&叶子节点的权重\\
     $\gamma$&控制叶子节点个数的惩罚权重\\
     $\lambda$&控制叶子节点的惩罚权重\\
\bottomrule%第三道横线
\end{tabular}
\end{table}  
